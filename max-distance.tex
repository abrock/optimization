\documentclass{scrbook}

\usepackage[ngerman]{babel}
\usepackage[utf8]{inputenc}
%\usepackage{fontspec}
\usepackage{amssymb, amsmath, microtype}
\usepackage{dsfont, xspace}
\usepackage{stmaryrd}
\usepackage{textcomp}
\usepackage{xcolor}
\usepackage[colorlinks]{hyperref}
\definecolor{darkblue}{HTML}{016394}
\definecolor{darkorange}{HTML}{B64E20}
\definecolor{darkgreen}{HTML}{518928}
\definecolor{darkyellow}{HTML}{917109}
\definecolor{darkpurple}{HTML}{421c5d}
\hypersetup{linkcolor=darkblue, citecolor=darkgreen, urlcolor=darkorange}
%\usepackage{romannum}
\usepackage[arrow, matrix, curve]{xy}

%\title{Numerische Mathematik 2}
%\author{Stephan Körkel}
%\date{\today}

\newcommand\dominiertvon{\prec}
\newcommand\dominiert{\succ}

\newcommand\Stab{\mathrm{Stab}}
\newcommand\kranzprodukt{\wr}
\newcommand\id{\mathrm{id}}
\newcommand\isomorph{\tilde =}
\newcommand\Aut{\mathrm{Aut}}
\newcommand\Out{\mathrm{Out}}
\newcommand\dist{\mathrm{dist}}
\newcommand\Inn{\mathrm{Inn}}
\newcommand\Gal{\mathrm{Gal}}
\newcommand\Isom{\mathrm{Isom}}
\newcommand\Hom{\mathrm{Hom}}
\newcommand\Tau{\mathcal T}
\newcommand\normal{\triangleleft}
\newcommand\wirkt{\to}
\newcommand\diam{\mathrm{diam}}
\newcommand\vol{\mathrm{vol}}

\newcommand\Bin{\mathrm{Bin}}
\newcommand\WKeit{Wahrscheinlichkeit\xspace}
\newcommand\hateq{\mathrel{\hat=}}
\newcommand\WRaum{Wahrscheinlichkeitsraum\xspace}
\newcommand\Rho{P}

\newcommand\obda{o.\,B.\,d.\,A.\xspace}
\newcommand \iid{i.\,i.\,d.\xspace}
\newcommand\Obda{O.\,B.\,d.\,A.\xspace}
\newcommand\zB{z.\,B.\xspace}
\newcommand\st{s.\,t.\xspace}
\newcommand\stm{ \text{ \st }\; }
\renewcommand\dh{d.\,h.\xspace}
\newcommand\Dh{D.\,h.\xspace}

\newcommand\mathstuff{\mathds}
\newcommand\bitm{\begin{itemize}}
\newcommand\eitm{\end{itemize}}
\newcommand\mus{\mu^*}
\newcommand\var{\mathrm{Var}}

\newcommand\that{{\hat t}}
\newcommand\tn{{t_0}}

\newcommand\notto{\nrightarrow}
\newcommand\notra{\nRightarrow}

\newcommand\prodl{\prod\limits}
\newcommand\xnin{(X_n)_{n\in\N}}
\newcommand\xninn{(X_n)_{n\in\N_0}}
\newcommand\nin{{n\in\N}}
\newcommand\ninn{{n\in\N_0}}

\newcommand\tensor{\oplus}
\newcommand\om{\omega}
\newcommand\Om{\Omega}

\newcommand\mA{\mathcal A}
\newcommand\mB{\mathcal B}
\newcommand\mD{\mathcal D}
\newcommand\mE{\mathcal E}
\newcommand\mF{\mathcal F}
\newcommand\mM{\mathcal M}
\newcommand\mN{\mathcal N}
\newcommand\mO{\mathcal O}
\newcommand\mI{\mathcal I}
\newcommand\mL{\mathcal L}
\newcommand\mX{\mathcal X}

\newcommand\Cay{\mathrm{Cay}}

\newcommand\one{\mathds 1}
\newcommand\bigcu{\bigcup\limits}
\newcommand\bigcui{\bigcup\limits_{i\in\N}}
\newcommand\bigcun{\bigcup\limits_{n\in\N}}
\newcommand\bigca{\bigcap\limits}
\newcommand\bigcai{\bigcap\limits_{i\in\N}}
\newcommand\bigcan{\bigcap\limits_{n\in\N}}
\newcommand\suml{\sum\limits}
\newcommand\sumi{\sum\limits_{i\in\N}}
\newcommand\sumn{\sum\limits_{n\in\N}}
\newcommand\sumk{\sum\limits_{k\in\N}}
\newcommand\intl{\int\limits}

\newcommand\Chi{\mathcal{X}}
\newcommand\cov{\mathrm{cov}}
\newcommand\vwahr{v_{wahr}}
\newcommand{\BIGOP}[1]{\mathop{\mathchoice%
{\raise-0.22em\hbox{\huge $#1$}}%
{\raise-0.05em\hbox{\Large $#1$}}{\hbox{\large $#1$}}{#1}}}
\newcommand{\bigtimes}{\BIGOP{\times}}
% nur fuer Bigboxplus andere Korrekturen
\newcommand{\BIGboxplus}{\mathop{\mathchoice%
{\raise-0.35em\hbox{\huge $\boxplus$}}%
{\raise-0.15em\hbox{\Large $\boxplus$}}{\hbox{\large $\boxplus$}}{\boxplus}}}

\newcommand\zv{Zufallsvariable\xspace}
\newcommand\vf{Verteilungsfunktion\xspace}

\renewcommand\phi{\varphi}
%\newcommand\ellipsen{\circledast}
\newcommand\ellipsen{\diamondsuit}
\newcommand\bpm{\begin{pmatrix}}
\newcommand\epm{\end{pmatrix}}
\newcommand\epsmach{\varepsilon_{mach}}
\newcommand\diag{\mathrm{diag}}
\newcommand\ov{\overline}
\newcommand\orthogonal{\perp}
\newcommand\Rang{\mathrm{Rg}}
\newcommand\Range{\mathrm{Range}}
\newcommand\trace{\mathrm{Sp}}
\newcommand\Image{\mathrm{Bild}}
\newcommand\Rg{\mathrm{Rg}}
\newcommand \tend{{t_{end}}}
\newcommand\Log{\mathrm{Log}}
\newcommand\R{\mathstuff{R}}
\renewcommand\P{\mathstuff{P}}
\newcommand\Q{\mathstuff{Q}}
\newcommand\C{\mathstuff{C}}
\newcommand\E{\mathstuff{E}}
\newcommand\N{\mathstuff{N}}
%\renewcommand\H{\mathstuff{H}}
\newcommand\D{\mathstuff{D}}
\newcommand\Z{\mathstuff{Z}}
\newcommand\F{\mathstuff{F}}
\newcommand\K{\mathstuff{K}}
\renewcommand\H{\mathstuff{H}}
\newcommand\Ho{\mathstuff{H}}
\newcommand\re{\mathrm{Re\,}}
\newcommand\im{\mathrm{Im\,}}
\newcommand\argmax{\arg\!\max}
\newcommand\argmin{\arg\!\min}
\renewcommand\l{\left}
\renewcommand\r{\right}
\newcommand\dt{\,\mathrm dt}
\newcommand\dd{\,\mathrm d}
\newcommand\dmu{\,\mathrm d \mu}
\newcommand\fue{f.\,ü.\xspace}
\newcommand\fs{f.\,s.\xspace}
\newcommand\RA{\Rightarrow}
\newcommand\LA{\Leftarrow}
\newcommand\LRA{\Leftrightarrow}
\newcommand\la\langle
\newcommand\ra\rangle
\newcommand\End{\mathrm{End}}

\newcommand\salgebra{$\sigma$-Algebra\xspace}
\newcommand\salgebren{$\sigma$-Algebren\xspace}
\newcommand\potenz{\mathcal P}
\newcommand\mP{\mathcal P}

\newcommand\roma{\begin{enumerate}\renewcommand{\labelenumi}{(\roman{enumi})}}
\newcommand\rome{\end{enumerate}}

\newcommand\bitmn{\begin{enumerate}}
\newcommand\eitmn{\end{enumerate}}


\newcommand\zb{z.\,B.\xspace}
\newcommand\Zb{Z.\,B.\xspace}

\newcommand\empha{\emph}

\newcommand\eps{\varepsilon}

\newcommand\chaptr[1]{\chapter*{Kapitel #1} \addcontentsline{toc}{chapter}{Kapitel #1}}

\newcommand\msection[1]{\section*{#1} \addcontentsline{toc}{section}{#1}}

\newcommand\msubsection[1]{\subsection*{#1} \addcontentsline{toc}{subsection}{#1}}
%\newcommand\msubsubsection[1]{\subsection*{#1} \addcontentsline{toc}{subsection}{:: #1}}

\newcommand\bemerkung[1]{\subsection*{Bemerkung #1}}

\newcommand\definition[1]{\subsection*{Definition #1} \addcontentsline{toc}{subsection}{Definition #1}}
\newcommand\satz[1]{\subsection*{Satz #1} \addcontentsline{toc}{subsection}{Satz #1}}
\newcommand\lemma[1]{\subsection*{Lemma #1} \addcontentsline{toc}{subsection}{Lemma #1}}
\newcommand\korollar[1]{\subsection*{Korollar #1} \addcontentsline{toc}{subsection}{Korollar #1}}
\newcommand\beweis[1]{\subsection*{Beweis #1}}
\newcommand\beispiel[1]{\subsection*{Beispiel #1} \addcontentsline{toc}{subsection}{Beispiel #1}}
\newcommand\beispiele[1]{\subsection*{Beispiele #1} \addcontentsline{toc}{subsection}{Beispiele #1}}
\newcommand\proposition[1]{\subsection*{Proposition #1} \addcontentsline{toc}{subsection}{Proposition #1}}

\newcommand\upto{\nearrow}
\newcommand\downto{\searrow}
\newcommand\qed{q.\,e.\,d.}

\newcommand\msubsubsection[1]{\subsubsection*{#1}}

\newcommand\omfp{(\Om, \mF,\mP)}
\newcommand\nfty{{n\to\infty}}

\newcommand\tow{\stackrel{\text{w}}\to}
\newcommand\tod{\stackrel{\text{d}}\to}
\newcommand\tonfty{\stackrel{n\to \infty}\longrightarrow}
\newcommand\toP{\stackrel{\P}\to}
\newcommand\tofs{\stackrel{\text{\fs}}\longrightarrow}

\begin{document}






Let $P_0, \cdots, P_3$ be the control points of curve $p$ and $P_0, Q_1, Q_2, P_3$
the control points of curve $q$.
We have $\|q(t)-p(t)\|$ over-estimates the true distance between the point $p(t)$ and the curve $q$,
since the distance $\dist(q(t), p)$ is defined by $\min_{t'} \|p(t) - q(t')\|$.
Therefore $\max_{t} \|p(t)-q(t)\|$ over-estimates $\max_t \dist(p(t), q)$.
%
\begin{align}
\min_{t'} \|p(t)-q(t')\| &\leq \|p(t)-q(t)\| \\
\RA \dist(p(t), q) &\leq \|p(t)-q(t)\| \\
\RA \max_t \dist(p(t), q) &\leq \max_t \|p(t)-q(t)\|
\end{align}

\section{Calculating $\argmax_t {\|p(t)-q(t)\|_2^2}$}

We calculate $\argmax_t {\|p(t)-q(t)\|_2^2}$:
%
\begin{align}
\|p(t)-q(t)\|_2^2 &= \|(3(1-t)^2 t P_1 + 3 (1-t) t^2 P_2) - (3(1-t)^2 t Q_1 + 3(1-t) t^2 Q_2)\|_2^2 \\
&= \|3(1-t)^2 t (P_1-Q_1) + 3 (1-t) t^2 (P_2 - Q_2)\|_2^2 \\
&= 9 t^2 (1-t)^2 \| (1-t) (P_1-Q_1) + t (P_2 - Q_2)  \|_2^2 \\
&= 9 t^2 (1-t)^2 \l\| \bpm (1-t) (P_{1x} - Q_{1x}) + t(P_{2x} - Q_{2x}) \\ (1-t) (P_{1y} - Q_{1y}) + t(P_{2y} - Q_{2y}) \epm \r\|_2^2
\end{align}
%
For ease of calculation we introduce variable substitutions:
%
\begin{align}
\alpha_{1x} :&= P_{1x} - Q_{1x} \\
\alpha_{2x} :&= P_{2x} - Q_{2x} \\
\alpha_{1y} :&= P_{1y} - Q_{1y} \\
\alpha_{2y} :&= P_{2y} - Q_{2y} \\
\RA \|p(t)-q(t)\|_2^2 &= 9 t^2 (1-t)^2 [ ((1-t)\alpha_{1x} + t \alpha_{2x})^2 + ((1-t)\alpha_{1y} + t \alpha_{2y})^2 ]
\end{align}
%
Let's calculate the first derivative wrt. $t$ and see what we get:
Well, we obviously get a polynomial of degree 3 and we need to find the roots which
is no fun, so lets's do something else: Triangle inequality.

\section{Using the triangle inequality}

We have $\|a+b\| \leq \|a\| + \|b\|$ for any two vectors $a,b$ and any norm $\|\cdot\|$.
Therefore:
%
\begin{align}
\|p(t)-q(t)\| &= \|(3(1-t)^2 t P_1 + 3 (1-t) t^2 P_2) - (3(1-t)^2 t Q_1 + 3(1-t) t^2 Q_2)\| \\
&\leq \|3(1-t)^2 t (P_1-Q_1)\| + \|3 (1-t) t^2 (P_2 - Q_2)\| =: \psi(t) \\
&= 3(1-t)^2 t \underbrace{\|P_1-Q_1\|}_{:=\alpha_1} + 3 (1-t) t^2 \underbrace{\|P_2 - Q_2\|}_{\alpha_2} \\
&= 3(1-t)^2 t \alpha_1 + 3 (1-t) t^2 \alpha_2 \\
\end{align}
%
Now we have a upper bound $\psi(t)$ for each $t \in [0,1]$.
We maximize it wrt. $t$ and obtain an upper bound for $\dist(p,q)$.
%
\begin{align}
0 &= \partial_t [3(1-t)^2 t \alpha_1 + 3 (1-t) t^2 \alpha_2] \\
\LRA 0 &= \partial_t [(1-2t+t^2)t \alpha_1 + (t^2-t^3) \alpha_2] \\
\LRA 0 &= \partial_t [(t-2t^2+t^3) \alpha_1 + (t^2-t^3) \alpha_2] \\
\LRA 0 &= (1 - 4t+3t^2) \alpha_1 + (2t-3t^2) \alpha_2 \\
\LRA 0 &= \alpha_1 + (2\alpha_2 - 4 \alpha_1)t + (3 \alpha_1 - 3 \alpha_2)t^2 \\
\end{align}
%
And that's just implementation stuff, calculate both roots $t_1, t_2$,
the upper bound then is $\max \{\psi(t_1), \psi(t_2)\}$.
$t_1$ and $t_2$ are probably good candidates for the actual time values
maximizing the distance between $p$ and $q$, therefore one could
calculate:
%
\begin{align}
l_1 :&= \min_{t'} \|p(t_1)-q(t')\| \\
l_2 :&= \min_{t'} \|p(t_2)-q(t')\| \\
l_3 :&= \min_{t'} \|p(t')-q(t_1)\| \\
l_4 :&= \min_{t'} \|p(t')-q(t_2)\| \\
\end{align}
%
Then $\max \{l_1, l_2, l_3, l_4\}$ is a lower bound for the maximum distance of $p$ and $q$.

\section{Estimation without finding roots of polynomials}

We want to do something even simpler.
We know now that:
%
\begin{align*}
\dist(p,q) &\leq \max_t [ 3(1-t)^2 t \alpha_1 + 3 (1-t) t^2 \alpha_2 ]
\end{align*}
%
We want to find the maximum value of both summands.
%
\begin{align*}
f_1(t) :&= (1-t)^2 t = (1-2t+t^2)t = t - 2t^2 + t^3 \\
f_1(1) &= f_1(0) = 0 \\
f_1'(t) &= 3t^2 -4t +1 \\
t_{1,2} :&= \frac{4 \pm \sqrt{16-12}}{6} = \frac {4 \pm 2}6 = \{1, 1/3\} \\
f_1(1/3) &= (2/3)^2 \frac 13 = \frac 49 \frac 13 = \frac{4}{27} \\
f_2(t) :&= (1-t)t^2 = f_1(1-t) \\
\RA f_2(2/3) &= \frac 4{27}
\end{align*}
%
Now we can derive a simpler, but coarser estimation:
%
\begin{align*}
\dist(p,q) &\leq \max_t  [ 3(1-t)^2 t \alpha_1 + 3 (1-t) t^2 \alpha_2 ] \\
&\leq \max_t 3(1-t)^2 t \alpha_1 + \max_t 3 (1-t) t^2 \alpha_2 \\
&= \frac 49 \alpha_1 + \frac 49 \alpha_2 = \frac 49 (\alpha_1 + \alpha_2) \\
&= \frac 49 (\|P_1-Q_1\| + \|P_2-Q_2\|)
\end{align*}

\section{Estimation using matrix norms}

For a matrix $A$ and vector $b$ we have $\|A b\|_2 \leq \|A\|_2 \|b\|_2$ where $\|A\|_2$ denotes the spectral norm of $A$.

We have:
%
\begin{align*}
\|p(t)-q(t)\|_2 &= \|3(1-t)^2 t (P_1-Q_1) + 3 (1-t) t^2 (P_2 - Q_2)\|_2 \\
&= \l\| \bpm 3(1-t)^2 t & 3(1-t)t^2 \epm \bpm P_1 - Q_1 \\ P_2 - Q_2 \epm \r\|_2 \\
&= 3 \l\| \underbrace{\bpm (1-t)^2 t & (1-t)t^2 \epm}_{=:A} \underbrace{\bpm P_1 - Q_1 \\ P_2 - Q_2 \epm}_{=: b} \r\|_2 \\
&\leq 3 \l\| \bpm (1-t)^2 t & (1-t)t^2 \epm \r\|_2 \l\| \bpm P_1 - Q_1 \\ P_2 - Q_2 \epm \r\|_2 \\
\end{align*}
%
For this special matrix the spectral norm is equal to the Euklidean norm of the row $A$, so we need to maximize:
%
\begin{align*}
f(t) :&= ((1-t)^2t)^2 + ((1-t)t^2)^2 
\end{align*}
%
We used a CAS to find that $f$ has only one local maximum in the interval $[0,1]$ at $t=1/2$
$f(1/2) = 1/32$, so $\|A\|_2 = \sqrt{1/32} = \sqrt{1/2}\sqrt{1/16} = 1/(4\sqrt 2)$.
Therefore:
%
\begin{align*}
\dist(p,q) &\leq \frac{3}{4\sqrt 2} \l\| \underbrace{\bpm P_1 - Q_1 \\ P_2 - Q_2 \epm}_{=b} \r\|_2
\end{align*}
%
Note that $b$ is in this case a $2\times 2$-matrix, so we need a estimation of its spectral norm.
Standard estimations are $\sqrt{\|b\|_\infty \cdot \|b\|_1}$ and $\|b\|_F$.

\end{document}
